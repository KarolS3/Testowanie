\documentclass[12pt,a4paper]{article}

\usepackage[utf8]{inputenc}
\usepackage[T1]{fontenc}
\usepackage[polish]{babel}
\usepackage{graphicx}
\usepackage{amsmath,amssymb}
\usepackage{booktabs}

\title{Mój pierwszy dokument LaTeX}
\author{Karol Stolc}
\date{\today}

\begin{document}

\maketitle

\section{Nagłówek}

To jest przykładowy tekst w dokumencie LaTeX.  
Przykład wzoru matematycznego w tekście: pole koła dane jest wzorem
$P = \pi r^2$, gdzie $r$ oznacza promień.

\subsection{Podnagłówek}

Poniżej znajduje się lista numerowana:

\begin{enumerate}
    \item Pierwszy element listy
    \item Drugi element listy
    \item Trzeci element listy
\end{enumerate}

A teraz lista punktowana:

\begin{itemize}
    \item Element pierwszy
    \item Element drugi
    \item Element trzeci
\end{itemize}

\subsection{Tabela}

Przykładowa tabela danych:

\begin{center}
\begin{tabular}{ccc}
\toprule
Kolumna 1 & Kolumna 2 & Kolumna 3 \\
\midrule
A & 10 & 20 \\
B & 30 & 40 \\
C & 50 & 60 \\
\bottomrule
\end{tabular}
\end{center}

\subsection{Wzory matematyczne}

Przykład wzoru zapisanego osobno:

\begin{equation}
\int_{0}^{1} x^2 \, dx = \frac{1}{3}
\end{equation}

Oraz równanie kwadratowe:

\begin{equation}
ax^2 + bx + c = 0
\end{equation}

\end{document}
